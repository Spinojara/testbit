% bitbit, a bitboard based chess engine written in c.
% Copyright (C) 2022-2024 Isak Ellmer
%
% This program is free software: you can redistribute it and/or modify
% it under the terms of the GNU General Public License, version 2 as
% published by the Free Software Foundation.
%
% This program is distributed in the hope that it will be useful,
% but WITHOUT ANY WARRANTY; without even the implied warranty of
% MERCHANTABILITY or FITNESS FOR A PARTICULAR PURPOSE.  See the
% GNU General Public License for more details.
%
% You should have received a copy of the GNU General Public License
% along with this program.  If not, see <https://www.gnu.org/licenses/>.

\documentclass{article}

\usepackage{amsmath}
\usepackage{amsthm}
\usepackage{hyperref}
\usepackage{cleveref}
\usepackage{url}

\theoremstyle{plain}
\newtheorem{theorem}     {\bf Theorem}     [section]
\newtheorem{proposition} {\bf Proposition} [section]
\newtheorem{corollary}   {\bf Corollary}   [section]
\newtheorem{lemma}       {\bf Lemma}       [section]
\theoremstyle{definition} 
\newtheorem{example}     {\bf Example}     [section]
\newtheorem{remark}      {\bf Remark}      [section]

\title{A Note on Estimating Elo}
\author{Isak Ellmer}
\date{December 2023}

\begin{document}
\maketitle

\section{Introduction}
Suppose that two players are playing a game of chess. We want to make some inference
on the difference in strength between the two players. Let $l$, $d$ and $w$ be the
probabilities that player $1$ loses, draws or wins respectively. It is not hard to see
that player $1$ is stronger than player $2$ if and only if $w>l$. In fact, $d$ does not
matter at all, if we only want to know which of the two players is stronger.

Consider however the two following cases. In the first case, $l=d=0$ and $w=1$. And in
the second case, $l=0$, $d=0.99$ and $w=0.01$. We see that player $1$ is stronger than
player $2$ in both of these cases, but much more so in the first.

To quantify this
strength difference we naturally look at the expected score of a game. If a loss is worth
$0$ points, a draw $\frac12$ and a win $1$, the expected score is given by
$s=\frac12d+w$. The closer $s$ is to $\frac12$ the more even the games are. And if $s$
is close to $1$ then player $1$ wins always every game. We see that in the first case
$s=\frac12\cdot0+1\cdot1=1$, and in the second case $s=\frac12\cdot0.99+1\cdot0.01=0.505$.

We see that player $1$ is stronger than player $2$ if and only if $s>\frac12$. This is
because $l+d+w=1$ and we then have
$$s=\frac12d+w=\frac12(d+w)+\frac12w=\frac12+\frac12(w-l).$$

The relationship between expected score and Elo difference \cite{wikipedia_elo} is given
by $$\Delta E=S^{-1}(s)$$ where $$S(x)=\frac1{1+10^{-x/400}}.$$ The inverse of $S$
is given by $$S^{-1}(s)=-\frac{400}{\log10}\log\left(\frac1s-1\right).$$

\section{Estimating Elo}\label{confidence}
We make the language a little more rigorous. Let $X$ be a random variable taking
the values $0$, $\frac12$ and $1$ with probabilities $l$, $d$ and $w$. Suppose that
$x_1,...,x_N$ are independent observations of $X$. The maximum likelihood estimator
$\hat s$ of $s=E(X)=\frac12d+w$ is given by $$\hat s=\frac1N\sum_{n=1}^Nx_n.$$
If $\sigma^2$ is the variance of $X$, then the central limit theorem
\cite{wikipedia_clt} asserts that $$\frac{\hat s-s}{\sigma/\sqrt{N}}$$ will approach
$N(0,1)$ in distribution.

We want to construct a confidence interval
for $s$ of confidence grade $1-\alpha$. Let $\lambda_{\alpha/2}$ be the upper $\alpha/2$
quantile of $N(0,1)$. Then $$P\left(-\lambda_{\alpha/2}<\frac{\hat s-s}{\sigma/\sqrt{N}}<
\lambda_{\alpha/2}\right)\approx1-\alpha.$$ Equivalently $$P\left(\hat s-
\frac{\lambda_{\alpha/2}\sigma}{\sqrt{N}}<s<\hat s+\frac{\lambda_{\alpha/2}\sigma}{\sqrt{N}}
\right)\approx1-\alpha.$$ Since $S$ is an increasing function we easily derive
the confidence interval
$$P\left(S^{-1}\left(\hat s-\frac{\lambda_{\alpha/2}\sigma}{\sqrt{N}}\right)<\Delta E<
S^{-1}\left(\hat s+\frac{\lambda_{\alpha/2}\sigma}{\sqrt{N}}\right)\right)\approx1-\alpha$$
for $\Delta E=S^{-1}(s)$. Note however that unless $\hat s=0$, this interval will not be
centered around $\Delta\hat E=S^{-1}(\hat s)$.

\subsection{Centered Approximate Confidence Intervals}
We will take two approaches. The first approach assumes that the Elo difference is small
and that $N$ is large. The second approach assumes only that $N$ is large.

We begin with the first approach. We have $s=S(\Delta E)$. Expanding $S$ in its first
order Taylor polynomial around $0$ implies $$s=S(\Delta E)\approx S(0)+\Delta ES'(0)=\frac12+
\Delta E\frac{\log10}{1600}.$$ Solving for $\Delta E$ then gives
$$\Delta E\approx\frac{1600}{\log10}\left(s-\frac12\right).$$
Starting with the confidence interval $$P\left(\hat s-\frac{\lambda_{\alpha/2}\sigma}{\sqrt{N}}
<s<\hat s+\frac{\lambda_{\alpha/2}\sigma}{\sqrt{N}}\right)\approx1-\alpha$$ from
\Cref{confidence}, we subtract $\frac12$ and multiply by the constant
$\beta=\frac{1600}{\log10}$. This gives
$$P\left(\beta\left(\hat s-\frac12\right)-\frac{\beta\lambda_{\alpha/2}\sigma}
{\sqrt{N}}<\beta\left(s-\frac12\right)<\beta\left(\hat s-\frac12\right)+
\frac{\beta\lambda_{\alpha/2}\sigma}{\sqrt{N}}\right)\approx1-\alpha.$$
The assumption that $\Delta E$ is small now gives $$P\left(\Delta\hat E-
\frac{\beta\lambda_{\alpha/2}\sigma}{\sqrt{N}}<\Delta E<\Delta\hat E+
\frac{\beta\lambda_{\alpha/2}\sigma}{\sqrt{N}}\right)\approx1-\alpha.$$

For the second approach we start from
$$P\left(S^{-1}\left(\hat s-\frac{\lambda_{\alpha/2}\sigma}{\sqrt{N}}\right)<\Delta E<
S^{-1}\left(\hat s+\frac{\lambda_{\alpha/2}\sigma}{\sqrt{N}}\right)\right)\approx1-\alpha.$$
We now instead expand $S^{-1}$ in its first order Taylor polynomial around $s=\hat s$.
We simply get $$S^{-1}\left(\hat s\pm\frac{\lambda_{\alpha/2}
\sigma}{\sqrt{N}}\right)$$$$\approx S^{-1}\left(\hat s\right)+\left(\hat s\pm\frac{\lambda_{\alpha/2}
\sigma}{\sqrt{N}}-\hat s\right)S^{-1}{'}\left(\hat s\right)$$$$=
\Delta\hat E\pm\frac{\lambda_{\alpha/2}\sigma}{\sqrt{N}}\frac1{S'\left(\Delta\hat E\right)}.$$
This implies $$P\left(\Delta\hat E-\frac{\lambda_{\alpha/2}\sigma}{\sqrt{N}}
\frac1{S'\left(\Delta\hat E\right)}<\Delta E<\Delta\hat E+\frac{\lambda_{\alpha/2}\sigma}{\sqrt{N}}
\frac1{S'\left(\Delta\hat E\right)}\right)\approx1-\alpha.$$

\bibliographystyle{plain}
\bibliography{\jobname}

\end{document}
